% You should title the file with a .tex extension (hw1.tex, for example)
\documentclass[11pt]{article}

\usepackage{amsmath}
\usepackage{amssymb}
\usepackage{fancyhdr}

\oddsidemargin0cm
\topmargin-2cm     %I recommend adding these three lines to increase the 
\textwidth16.5cm   %amount of usable space on the page (and save trees)
\textheight23.5cm  

\newcommand{\question}[2] {\vspace{.25in} \hrule\vspace{0.5em}
\noindent{\bf #1: #2} \vspace{0.5em}
\hrule \vspace{.10in}}
\renewcommand{\part}[1] {\vspace{.10in} {\bf (#1)}}

\newcommand{\myname}{Write Yr Name here!}
\newcommand{\myandrew}{write-yr-andrew-id-here@andrew.cmu.edu}
\newcommand{\myhwnum}{homework-number-here}

\pagestyle{fancyplain}
\lhead{\fancyplain{}{\textbf{HW\myhwnum}}}      % Note the different brackets!
\rhead{\fancyplain{}{\myname\\ \myandrew}}
\chead{\fancyplain{}{15-251}}

\begin{document}

\medskip                        % Skip a "medium" amount of space
                                % (latex determines what medium is)
                                % Also try: \bigskip, \littleskip

\thispagestyle{plain}
\begin{center}                  % Center the following lines
{\Large 15-251 Assignment \myhwnum} \\
\myname \\
\myandrew \\
Collaborators: no one / hbovik,johnq,fhacker\\
Your recitation section \\
The date \\
\end{center}

\question{1}{Mathematical Symbols}

This is an example of an answer to a homework question.  In your
answer, you may want to use a variety of mathematical symbols:

% this is one way to make a list. Another would be to say
% \begin{enumerate} and \end{enumerate}, which would give numbers instead
% of bullets to the items
\begin{itemize}
\item Fractions: $\frac{2}{3}$, or (equivalently) ${2 \over 3}$
\item Binomial coefficients: ${5 \choose 2} = 10$
\item Subscripts and superscripts: $t_0$, $t^2$, $t_0^{2 \over 3}$, 
\item Greek letters: $\alpha$, $\beta$, $\gamma$, $\lambda$, $\Pi$.
\item Summations: $\sum_{i=1}^n i = {n(n+1) \over 2}$.
\end{itemize}

You can refer to Leslie Lamport's ``\LaTeX\ User's Guide and
Reference Manual'' for more useful info on mathematical typesetting
with \LaTeX.  Pages 42-46 outline many of the useful math symbols
and functions.

Another good reference for the mathematical symbols of \LaTeX\ is 
``Inessential LaTeX'' by the MIT SIPB group.  Find it on the Web at
\verb|http://www.mit.edu/afs/sipb/project/doc/latex/guide.PS|


\question{2}{Little Gauss's Formula}

This is another example of a question.  In this case, it's
a multi-part question.

\part{a} Recall {\em Little Gauss's formula}:

% equations are automatically centered!
\begin{eqnarray}
\sum_{i=1}^n i = {{n(n+1)} \over 2}  % as usual, liberal use of curly braces...
\label{little-gauss}        % labelled so we can refer to this formula by number
\end{eqnarray}

\part{b} Now, equation \ref{little-gauss} can be proven by induction as
follows:

\begin{itemize}
\item {\bf Base case}: $n=1$: $1 = 1(2)/2=1$.
\item {\bf Inductive hypothesis}: assume the equation holds for $n=2...k$.
\item {\bf Inductive step}: for $n=k+1$, we have
\begin{eqnarray*}    % the star suppresses the equation numbers 
\sum_{i=1}^{k+1} i = (k+1) + \sum_{i=1}^k i
\end{eqnarray*}
Using the inductive hypothesis, we can substitute for the second term
on the righthand side:
\begin{eqnarray*}
\sum_{i=1}^{k+1} i &=& (k+1) + k(k+1)/2\\    
                   &=& {{2k+2 + k(k+1)} \over 2}\\
                   &=& {{k^2 + 3k + 2} \over 2}\\
                   &=& {{(k+1)(k+2)} \over 2}
\end{eqnarray*}
% notice \\ to indicate newline and the &=& to line up the equals signs 
\end{itemize}

Lo and behold! The last line shows that for $n=k+1$, little Gauss'
formula still holds for $n=k+1$! We've showed that the formula holds
for $n=1$, and we've shown that if it holds for $n=k$ it must hold for
$n=k+1$. Therefore, it must hold for all $n$.

\end{document}

